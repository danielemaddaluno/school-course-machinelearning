
%============================================================
%============================================================
%============================================================
\section[Derivato 1]{Derivato di tre sottostanti con barriere}

%------------------------------------------------------------
%------------------------------------------------------------
%------------------------------------------------------------

\subsection[Il dataset]{Il dataset}
\begin{frame}
	
	\frametitle{0.1 Il Dataset}
	
	\begin{block}{Il dataset ``serie\_storiche.xlsx'':}

		Il dataset contenuto in ``serie\_storiche.xlsx'' comprende i prezzi giornalieri di chiusura di mercato per 7 titoli azionari per il periodo che va dal 18/Feb/2010 al 14/Feb/2011.\\
		
		\begin{itemize}
			\item EXXON MOBIL CORP
			\item GENERAL ELECTRIC CO
			\item CITIGROUP INC
			\item MICROSOFT CORP
			\item BANK OF AMERICA CORP
			\item BP PLC	
			\item HSBC HOLDINGS PLC
		\end{itemize}

		
	\end{block}

\end{frame}


\begin{frame}
	
	\begin{block}{Tre azioni del dataset costituiscono i sottostanti del derivato preso in considerazione:}
		\begin{itemize}
			\item Sottostante 1: CITIGROUP INC
			\item Sottostante 2: BANK OF AMERICA CORP
			\item Sottostante 3: BP PLC
		\end{itemize}

	\end{block}
	
	\begin{figure}[!htbp]
		\centering
		\includegraphics[width=11cm]{images/99_old_misuraca_trash/assets/dataset.png}
	\end{figure}
\end{frame}


\subsection[I rendimenti]{I rendimenti}
\begin{frame}

	\frametitle{0.2 I rendimenti}
	
	\begin{block}{Calcolo dei rendimenti logaritmici giornalieri:}
		Indichiamo con $C_{i,t}$ il dato disponibile del prezzo di chiusura di mercato della azione $i$ in data $t$.\\
		Calcoliamo i rendimenti giornalieri logaritmici per ognuno dei tre asset, inserendo i risultati all'interno di una tabella che chiamiamo ``Returns''.
		$$R_{i,t} = log(C_{i,t}/C_{i,t-1})$$
		
	\end{block}
	
	\begin{figure}[!htbp]
		\centering
		\includegraphics[width=5cm]{images/99_old_misuraca_trash/assets/returns.png}
	\end{figure}
\end{frame}
	

\subsection[Le volatilità storiche e le barriere]{Le volatilità storiche e le barriere}
\begin{frame}
	\frametitle{0.3 Le volatilità e le barriere}

	\begin{block}{Calcolo delle volatilità storiche e delle barriere:}
		Calcoliamo la \textcolor{red}{volatilità} per ogni titolo $i$ a partire dai ritorni come segue:
		$$Vol\_i = STDEV(Returns[nome\_titolo\_i])*SQRT(252)$$
		Calcoliamo le \textcolor{red}{barriere} per ogni titolo $i$ a partire dai prezzi attuali:
		$$Threshold\_i = Cella\_S0_{i}*Cella\_Barrier$$
	\end{block}
	
	\begin{figure}[!htbp]
		\centering
		\includegraphics[width=4.5cm]{images/99_old_misuraca_trash/Subsection1Part1/options.png}
	\end{figure}

\end{frame}

\subsection[Pricing semplice]{Pricing semplice}
\begin{frame}
	\frametitle{1. Il pricing semplice con Bonus = 20 e Barrier = 140\%}

	\begin{block}{La simulazione Monte Carlo:}
		Per calcolare il prezzo del derivato considerato sono state scritte quattro funzioni:
		\begin{itemize}
			\item \textbf{derivative\_price}: al suo interno lancia un numero significativo di simulazioni (pari a n\_sim) al termine delle quali calcola il prezzo come una media scontata dei payoff a scadenza. 
			\item \textbf{derivative\_simulation}: lancia una singola simulazione dell'evoluzione dei prezzi dei sottostanti. Al termine restituisce una serie di informazioni tra le quali il prezzo del derivato a scadenza.
			\item \textbf{norm}: si occupa di generare campioni da una distribuzione normale con una certa media e deviazione standard. Viene utilizzata all'interno della funzione derivative\_simulation per generare degli shocks sui prezzi per ogni step di campionamento. 
			\item \textbf{boxM}: genera numeri random provenienti dalla distribuzione $\mathcal{N}(0, 1)$.
		\end{itemize}
	\end{block}

\end{frame}



\begin{frame}{Gestione della correlazione}
\begin{block}

Per gestire la correlazione tra i diversi sottostanti si è fatto quanto segue:
\begin{enumerate}

\item Stima della matrice di varianza-covarianza ($\Sigma$)
\item Decomposizione della matrice $\Sigma$ con Cholesky ($L$)
\item Simulazione dei sottostanti come $ d\textbf{W}^{*} = Ld\textbf{W} $ dove $dW_i= \sigma_i \cdot dZ_i$
\end{enumerate}

$$
L=
\begin{pmatrix} 1 & 0 & 0 \\  
\rho_{1,2} & \sqrt{1 - \rho^2_{1,2}} & 0 \\ 
\rho_{1,3} & \sqrt{\rho^2_{2,3} - \rho^2_{1,3}} &\sqrt{1 - \rho^2_{2,3}} \end{pmatrix}
$$

$$ 
\begin{cases} dS_{1} = S_{1} e^{(r - \frac{\sigma_{1}^2}{2})\Delta t + dW_{1}} \\ 
dS_{2} = S_{2} e^{(r - \frac{\sigma_{2}^2}{2})\Delta t + \rho_{1,2}dW_{1} + \sqrt{1 - \rho^2_{1,2}} dW_{2}}  \\ 
dS_{3} = S_{3} e^{(r - \frac{\sigma_{3}^2}{2})\Delta t + \rho_{1,3}dW_{1} + \sqrt{\rho^2_{2,3} - \rho^2_{1,3}} dW_{3} \sqrt{1 - \rho^2_{2,3}} dW_{3}}  
\end{cases}
$$

\end{block}
\end{frame}





\begin{frame}{Estratto di \color{black}\Verb+Es\_1\_Part\_1+}
	\inputminted[fontsize=\tiny]{vb.net}{images/99_old_misuraca_trash/assets/part1.txt}
\end{frame} 

\begin{frame}{Estratto della funzione \color{black}\Verb+derivative\_price+}
	\inputminted[fontsize=\tiny]{vb.net}{images/99_old_misuraca_trash/assets/derivative_price.txt}
\end{frame}

\begin{frame}{Estratto della funzione \color{black}\Verb+derivative\_simulation+}
	\inputminted[fontsize=\tiny]{vb.net}{images/99_old_misuraca_trash/assets/derivative_simulation.txt}
\end{frame}

\begin{frame}{Estratto della funzione \color{black}\Verb+derivative\_simulation+}
	\inputminted[fontsize=\tiny]{vb.net}{images/99_old_misuraca_trash/assets/derivative_simulation1.txt}
\end{frame}



\begin{frame}
	\begin{block}{Prezzo del derivato:}
		Dati gli input mostrati in figura in verde chiaro, il prezzo del derivato calcolato tramite simulazione Monte Carlo risulta essere pari a $\mathbf{\sim10.61}$\textbf{\textcurrency}.
	\end{block}
	 
	\begin{figure}[!htbp]
		\centering
		\includegraphics[angle=0,width=\linewidth]{images/99_old_misuraca_trash/Subsection1Part1/pricing_simple.png}
	\end{figure}
\end{frame}

\subsection[Pricing al variare del Bonus (K)]{Pricing al variare del Bonus (K)}
\begin{frame}
	\frametitle{2. Il pricing al variare del Bonus (K)}
	\begin{block}{Prezzo del derivato al variare del Bonus (K):}
		Dati gli input mostrati in figura in verde chiaro, al variare del Bonus (K) da 2 a 60 con un passo di 2, l'evoluzione del prezzo del derivato calcolato tramite simulazione Monte Carlo è mostrata in figura.
		Si può osservare che al crescere del Bonus (K) il prezzo tende ad aumentare.
	\end{block}
	
	\begin{figure}[!htbp]
		\centering
		\includegraphics[width=9cm]{images/99_old_misuraca_trash/Subsection1Part2/pricing_K.png}
	\end{figure}
\end{frame}

\begin{frame}{Estratto di \color{black}\Verb+Es\_1\_Part\_2+}
	\inputminted[fontsize=\tiny]{vb.net}{images/99_old_misuraca_trash/assets/part2.txt}
\end{frame} 

\begin{frame}{Estratto della funzione \color{black}\Verb+plotChart+}
	\inputminted[fontsize=\tiny]{vb.net}{images/99_old_misuraca_trash/assets/plotChart.txt}
\end{frame}
 
\subsection[Pricing al variare del livello della barriera]{Pricing al variare del livello della barriera}
\begin{frame}
	\frametitle{3. Il pricing al variare della barriera}

	\begin{block}{Prezzo del derivato al variare della barriera:}
		Dati gli input mostrati in figura in verde chiaro, per le 5 barriere prese in considerazione (da 110\% a 150\% con un passo del 10\%), si è calcolata l'evoluzione del prezzo del derivato e il numero di occorrenze di ciascuno dei tre eventi bonus.
	\end{block}

	\begin{figure}[!htbp]
		\centering
		\includegraphics[width=5cm]{images/99_old_misuraca_trash/Subsection1Part3/options.png}
		%\caption{ENEL Histfit Normale}
		%\label{Enel_HistFit_Normal} 
	\end{figure}

\end{frame}

\begin{frame}{Estratto di \color{black}\Verb+Es\_1\_Part\_3+}
	\inputminted[fontsize=\tiny]{vb.net}{images/99_old_misuraca_trash/assets/part3.txt}
\end{frame} 


\begin{frame}
	\begin{columns}
	
		\column{0.5\linewidth}
		\begin{figure}[!htbp]
			\centering
			\includegraphics[angle=0,width=\linewidth]{images/99_old_misuraca_trash/Subsection1Part3/barriers.png}
			%\caption{ENEL Histfit Normale}
			%\label{Enel_HistFit_Normal} 
		\end{figure}
		
		\column{0.5\linewidth}
		\begin{figure}[!htbp]
			\centering
			\includegraphics[angle=0,width=\linewidth]{images/99_old_misuraca_trash/Subsection1Part3/prices.png}
			%\caption{ENEL QQ-Plot Normale}
			%\label{Enel_QQ_Plot_Normal} 
		\end{figure}
	
	\end{columns} 
\end{frame}

\begin{frame} 

	\begin{block}{Percentuale evento bonus al variare della barriera:}
		Si può quindi osservare come al variare della barriera cambi la probabilità associata al verificarsi di ciascuno dei tre eventi bonus.\\
		Le due figure mostrate rappresentano questo concetto attraverso due tipologie di grafico distinte.
	\end{block}

	\begin{columns}
	
		\column{0.5\linewidth}
		\begin{figure}[!htbp]
			\centering
			\includegraphics[angle=0,width=\linewidth]{images/99_old_misuraca_trash/Subsection1Part3/proportions_1.png}
			%\caption{ENEL Histfit Normale}
			%\label{Enel_HistFit_Normal} 
		\end{figure}
		
		\column{0.5\linewidth}
		\begin{figure}[!htbp]
			\centering
			\includegraphics[angle=0,width=\linewidth]{images/99_old_misuraca_trash/Subsection1Part3/proportions_2.png}
			%\caption{ENEL QQ-Plot Normale}
			%\label{Enel_QQ_Plot_Normal} 
		\end{figure}
	
	\end{columns}
\end{frame}
