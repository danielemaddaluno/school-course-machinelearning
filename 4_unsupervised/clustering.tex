\subsection[Il Clustering]{Il Clustering}

\begin{frame}
	
	\frametitle{Apprendimento non supervisionato: il clustering}
	
	\begin{block}{Il clustering}
		Il \textbf{clustering} o \textbf{analisi dei gruppi} è un insieme di tecniche di analisi multivariata dei dati volte alla \textbf{selezione} e \textbf{raggruppamento} di \textbf{elementi omogenei} in un insieme di dati.
		\newlinedouble
		Le tecniche di clustering si basano su misure relative alla somiglianza tra gli elementi. In molti approcci questa similarità, o meglio, dissimilarità, è concepita in termini di \textbf{distanza in uno spazio multidimensionale}.
		\newlinedouble
		La bontà delle analisi ottenute dagli algoritmi di clustering dipende molto dalla scelta della \textbf{metrica}, e quindi da come è calcolata la distanza. Gli algoritmi di clustering raggruppano gli elementi sulla base della loro distanza reciproca, e quindi l'appartenenza o meno a un insieme dipende da quanto l'elemento preso in esame è distante dall'insieme stesso.
	\end{block}

\end{frame}


\subsubsection[Applicazioni Varie]{Applicazioni Varie}
\begin{frame}
	
	\frametitle{Il clustering: applicazioni}
	
	\begin{block}{Applicazioni Varie}
		Il clustering ha una miriade di utilizzi in una varietà di settori.\\
		Alcune \textbf{applicazioni comuni} per il clustering includono quanto segue:
		\begin{itemize}
			\item segmentazione del mercato
			\item customer profiling dei social network (individuando utenti simili)
			\item raggruppamento dei risultati di ricerca
			\item segmentazione delle immagini
			\item rilevamento delle anomalie (ad esempio, transazioni anomale)
		\end{itemize}
		
		Dopo la clusterizzazione, a ogni cluster viene assegnato un numero chiamato \textbf{cluster-ID}.
		%Ora puoi condensare l'intero set di funzionalità per un esempio nel suo ID cluster.\\
		Riuscire a rappresentare un esempio complesso mediante un semplice ID  rende il clustering uno strumento potente.
	\end{block}
		
\end{frame}


\begin{frame}
	
	\frametitle{Il clustering: applicazioni di supporto al Machine Learning}
	
	\begin{block}{Applicazioni di supporto}
		Estendendo l'idea, il clustering può semplificare datasets di grandi dimensioni.
		Alcune applicazioni di supporto al Machine Learning sono:
		\begin{itemize}
			\item \textbf{generalizzazione}: quando alcuni esempi in un cluster presentano features mancanti, è possibile ricavarli da altri esempi nel cluster
			\item \textbf{compressione}: le features degli esempi in un cluster possono essere sostituite dall'ID cluster pertinente. Questo semplifica le features e consente di risparmiare spazio di archiviazione.
			\item \textbf{privacy}: è possibile preservare la privacy raggruppando gli utenti e associando i dati degli utenti ad un cluster-ID invece che a utenti specifici. Per garantire che non sia possibile associare i dati utente a un utente specifico, il cluster deve raggruppare un numero sufficiente di utenti
		\end{itemize}
	\end{block}
	
\end{frame}