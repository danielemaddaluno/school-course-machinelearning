\subsection[Il numero ottimale di clusters $K$]{Il numero ottimale di clusters $K$}
\begin{frame}
	
	\frametitle{Algoritmi di Clustering: la scelta del numero di clusters $K$}
	
	%\begin{block}{Determining Optimal Clusters $K$}
		Non abbiamo trattato in alcun modo il problema di come effettuare una \textbf{scelta} oculata del \textbf{numero di clusters} $\pmb{K}$. Nei procedimenti affrontati abbiamo sempre dato per scontato che siamo noi a fissare questo valore $K$ (in realtà per il mean-shift fissiamo il bandwidth, il quale a sua volta determina il numero di clusters, ma il problema della scelta persiste).
		\newlinedouble
		Per aiutare l'analista, esistono diversi metodi numerici per determinare il \textbf{numero di cluster ottimale} $\pmb{K}$, di seguito elenchiamo i tre più diffusi:
		\begin{itemize}
			\item Elbow method
			\item Silhouette method
			\item Gap statistic
		\end{itemize}
		
		Per una descrizione dettagliata dei tre metodi leggere:\\
		\underline{\href{https://uc-r.github.io/kmeans_clustering\#optimal}{Determining Optimal Clusters}},	\href{https://scholar.google.com/citations?user=Fz5g0gcAAAAJ}{Bradley Boehmke}
	%\end{block}
	
\end{frame}
