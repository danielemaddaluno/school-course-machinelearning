\subsection[Creare proprie features]{Creare proprie features}

\begin{frame}
	
	\frametitle{{\color{GradientDescentDiagramRed}Creare proprie features}}

	%\begin{block}{}
		A volte i dati grezzi non dispongono delle feature sufficienti a svolgere operazioni di machine learning. In questi casi, vanno create delle nuove features. La \textbf{creazione di una feature} non significa creare dati dal nulla.
		\newlinedouble
		Per esempio, se state modellando il prezzo delle proprietà immobiliari, la superficie della proprietà ha ottime qualità predittive, perché le proprietà più grandi tendono a essere più costose; se invece della superficie fornite all’algoritmo di apprendimento la lunghezza dei lati della proprietà (le coordinate di latitudine e longitudine degli angoli), l’algoritmo potrebbe non saper che farsene delle informazioni che gli avete passato (almeno per la maggior parte degli algoritmi è così).
		\newlinedouble
		L'obiettivo è \textbf{creare nuove feature derivanti da quelle che già esistono} in modo che le nuove abbiano \textbf{capacità predittiva maggiore} rispetto a quelle originali (spesso implica conoscere bene il problema).
	%\end{block}
	
\end{frame}